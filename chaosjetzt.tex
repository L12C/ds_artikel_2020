\begin{DSarticle}[
    title={chaos.jetzt},
    author=
{
dc7ia=lektorat@dc7ia.eu,
ruru4143=ruru-cj-ds@r3.at,
raketenlurch=raketenlurch@riseup.net,
L12C=l12c@l12c.eu
}

So meinte ich das, aber sollte ansonnsten auch passen denke ich
    head=chaos.jetzt,
    tocentry=chaos.jetzt,
]
\begin{abstract}
Gibt es viele Jugendliche im Chaos?
Wer auf den verschiedenen Veranstaltungen den Blick schweifen lässt, wird zu dem Ergebnis gekommen sein:
Naja, es gibt schon einige.
Und dennoch findet man kaum kaum Zusammenschlüsse und in den meisten Räumen sind die Jugendlichen weit in der Unterzahl.
Diese Beobachtung teilten Teilnehmer:innen des Regiowochenende 2019 und machten sich daran etwas ändern\ldots
\end{abstract}

%~ \section{Unterüberschrift}

Auf dem CCCamp19 haben wir eine Session mit dem Namen \enquote{Junges Chaos bildet Banden} organisiert, um mit am Austausch interessierten
jungen Leuten zu überlegen, welche Strukturen wir uns wünschen.
Das Treffen hatte einigen Zulauf, was sicher nicht nur an den poolgekühlten Getränken lag, und brachte junge Chaot:innen mit Ideen und Tatendrang zusammen.
Wir fackelten nicht lange, legten uns Domain und Server zu und begannen mit dem Aufbau. Das Ergebnis ist die Gruppe \emph{chaos.jetzt}.

chaos.jetzt richtet sich in erster Linie an technische interessierte 14- bis 18-Jährige, die nicht weiter alleine vor sich hin basteln möchten und eine Community in ihrer Altersgruppe suchen.
Unser Ziel ist: junge Leute im Chaosumfeld gezielt zusammen bringen und vernetzen.
Wir wollen Projektideen realisieren, die eventuell allein nicht umsetzbar wären, und den Austausch über diese Projekte und weitere Ideen ermöglichen.
So wollen wir dafür sorgen, dass sich Jugendliche im Chaos insgesamt willkommener fühlen und schneller Anschluss an Gleichaltrige finden.

Unsere Kommunikation läuft dabei hauptsächlich über unseren Matrix-Chat.
Dort heißen wir natürlich auch Ältere willkommen, die sich für chaos.jetzt interessieren, Kontakte zu jüngeren Generationen suchen und diese unterstützen wollen.

Auf dem 36c3 sind wir dann als Gruppe das erste Mal auf einer Veranstaltung aufgetreten. Wir organisierten in der \emph{WikipakaWG} zusammen mit Jugend~hackt eine Assembly mit eigenem Dome.
Zu fast jeder Tages- und Nachtzeit machten es sich einige von uns auf den Sitzsäcken bequem und nutzten die Gelegenheit für Vernetzungs- und Diskussionsrunden sowie Workshops von und mit Jugendlichen.

Ende Februar 2020 hat unser erstes, selbstorganisierten Geekend stattgefunden. Wir waren in Göttingen im Neotopia zu Gast und haben 
uns in kleineren und größeren Gruppen über verschiedene Themen ausgetauscht. Da es das erste Treffen war, ging es inhaltlich mehr um Meta-Themen,
wie zum Beispiel \enquote{Wie wollen wir Entscheidungen treffen und Menschen mandatieren?}, \enquote{Was wollen wir inhaltlich machen?} und \enquote{Welche Tools 
wollen wir benutzen?}. In längeren Plena, die erstaunlich kurzweilig blieben, kamen wir zu ersten Ergebnissen. Dies bot eine gute Grundlage für weitere Diskussionen. Im Laufe des
	Wochenendes wurde \enquote{Wir wollen kaputte Strukturen im CCC reparieren - mit und ohne Lötkolben} zum neuen, inoffiziellen Motto. Neben der 
Arbeit sind natürlich auch gemütliches Zusammensitzen und mehrere Flauschhaufen mit Menschen und einer größeren Menge Plüschtieren 
(hauptsächlich Einhörner und Blåhajs) nicht zu kurz gekommen. Am Ende war es so schön, dass zur Abmilderung der allseits bekannten
Post-Chaos-Event-Depression direkt angefangen wurde das nächste Treffen zu planen.

Wenn ihr mitmachen wollt und/oder Infos zu uns, unserem Matrix-Server und unseren Events haben wollt schaut mal unter \url{https://chaos.jetzt/}.


%~ \DSBild[Quelle/Urheber]{bilder/pixel}
% Plakat „Junges Chaos bildet Banden“ vom Camp 2019
% Foto von Hernani via Twitter https://pbs.twimg.com/media/ECpCnN1U0AAw78x?format=jpg&name=900x900
% Bildbeschreibung:
%	Plakat an Holzmast beim Camp 2019 in Mildenberg. Überschrift: „Junges Chaos bildet Banden“. Hauptmotiv: Nach oben gestreckte, geballte linke Faust.
%	Klein am unteren Rand: Was: Wir wollen Junge Leute im Chaos Umfeld vernetzen.
%	Session: https://events.ccc.de/camp/2019/wiki/Session:Junges_Chaos_Bildet_Banden
%	Wann: Tag 3 – 16:00 Uhr (bis 17:30) \[23.08.2019\]
%	Wo: Workshop Zelt der Haecksen

%~ \begin{DSreferenzen}
	%~ \bibitem{Tiefstand}
	%~ Nachname, Vorname: \enquote{Titel} (Jahr), \url{url}

%~ \end{DSreferenzen}

\end{DSarticle}
